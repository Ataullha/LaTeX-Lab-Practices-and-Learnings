\documentclass[a4paper,12pt]{article}

\begin{document}

\begin{flushleft}
\textsl{\textbf{Md Ataullha, Reg 2018331081}}\\
Technical Writing and Presentation\\
Article Review, 3 November 2022
\end{flushleft}

\date{}
\title{Natural language processing: an introduction}
\author{Prakash M Nadkarni,Prakash M Nadkarni,Wendy W Chapman}
{\let\newpage\relax\maketitle} % title will not start from another page
\maketitle

\pagenumbering{arabic}

\section*{Summary}
An overview of the principles of Natural Language Processing is provided in this paper. It also explains the historical development of Natural Language Processing from the beginning to a modern natural language processing system and its applicability in the medical field. In the past, People used handwritten rules for Natural Language Processing. The limitations of handwritten rules are solved using statistical Natural Language Processing. Here, the subproblems of Natural Language Processing are divided into two levels, low and high, and the barriers to their use in the clinical field are mentioned. A comparison between generative and discriminative Machine learning models in Natural Language Processing is described and an overview of two generative models - Hidden Markov models, N-grams and two discriminative models Support vector machines, Conditional random fields models are discussed. The three pitfalls of IBM Watson in the field of medical diagnosis are highlighted, along with the future of modern Natural Language Processing. This research also emphasizes the difficulties in commodifying toolkits for Natural Language Processing.

\section*{Strong Points}

\begin{enumerate}
\item The potential features of Natural Language processing in the medical sector are highlighted.
\item The progression of Natural Language Processing from its roots to its modern form is highlighted clearly.
\item Clear descriptions of Natural Language Processing difficulties.
\item The potential pitfalls of using the IBM Watson model in the medical field are clearly explained.
\end{enumerate}

\section*{Weak Points}

\begin{enumerate}
\item Natural Language Processing concepts outside of the medical area are not discussed.
\item A small number of examples are used to describe Natural Language Processing sub-categories and ML models.
\item No statistical data points are used to compare four Machine learning models.
\end{enumerate}

\section*{Comments/Suggestions}
This paper is crucial for understanding the fundamental processes and development of Natural Language Processing for new scholars in the field. Despite its shortcomings, this is a fantastic resource for learning about natural language processing.
\end{document}