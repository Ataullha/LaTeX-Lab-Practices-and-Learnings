\documentclass[a4paper,12pt]{report}

\usepackage{color}
\usepackage{graphicx}
\usepackage{amsmath}
\usepackage{subcaption}
\usepackage{natbib}
\usepackage{url}

\setcounter{chapter}{1}

\begin{document}

\title{My First LaTeX Document}
\author{Md Ataullha}
\date{\today}
\maketitle

\pagenumbering{roman}
\tableofcontents
\listoftables
\listoffigures
\newpage
\pagenumbering{arabic}

\addcontentsline{toc}{chapter}{Practical 1: Document Structure}
\chapter*{Practical 1: Document Structure}

\section{Introduction}This is the Introduction section.

\section{Methods} 
This is the Methods section.

\subsection{Stage 1}
\label{sec1}
This is the Stage 1 subsection and there is a label on it.
\subsubsection{Part 1}
This is another subsubsection under subsection.


\subsection{Stage 2}
This is anoter Stage 2 subsection.

\section{Results}
Here is the results section. Referring to section \ref{sec1} on page \pageref{sec1}.

\paragraph{First Paragraph}
This is a simple paragraph named First Paragraph. We used the pragraph command here.
\subparagraph{First Paragraph Sub Paragraph}
This is a subparagraph under the First paragraph.

\stepcounter{chapter} % if we do \stepcounter{chapter}{2} it will produce extra 2
\addcontentsline{toc}{chapter}{Practical 2: Typesetting Text}
\chapter*{Practical 2:Typesetting Text }

\section{Font Effects}
\textit{textit command for italitcs font effect.}\\
\textsl{textsl command for slanted font effect.}\\
\textsc{textsc command for smallcaps font effect.}\\
\textbf{textbf command for bold font effect.}\\
\texttt{texttt command for teletype font effect.}\\
\textsf{textsf command for sans serif font effect.}\\
\textrm{textrm command for roman font effect.}\\
\underline{underline command for underline font effect.}\\
\underline{\textbf{underline and textsc command for underline and bold font effect.}}\\

\section{Color Effects}
{\color{red}This is Red.}\\
{\color{green}This is Green.}\\
{\color{blue}This is Blue.}\\
{\color{cyan}This is Cyan.}\\
{\color{magenta}This is Magenta.}\\
{\color{yellow} This is Yellow.}\\
{\color{white}This is White.}\\

\section{Size Effects}
{\tiny tiny command for tiny font size.}\\
{\scriptsize scriptsize  command for scriptsize font size.}\\
{\small small command for small font size.}\\
{\normalsize normalsize command for normalsize font size.}\\
{\large large command for large font size.}\\
{\Large Large command for Large font size.}\\
{\LARGE LARGE  command for LARGE font size.}\\
{\huge huge command for huge font size}\\

\section{Lists}
A list and sub-list with enumerate only \\
\begin{enumerate}
\item First item.
\item Second item.
\begin{itemize}
\item A sub-item.
\item Another sub-item.
\end{itemize}
\item Third item.
\end{enumerate}

A list and sub-list \\
\begin{itemize}
\item First item.
\item Second item.
\begin{itemize}
\item A sub-item.
\item Another sub-item.
\end{itemize}
\item Third item.
\end{itemize}

A custom type list\\
\begin{itemize}
\item[-] First item.
\item[+] Second item.
\begin{itemize}
\item[ABC] A sub-item.
\item[DEF] Another sub-item.
\end{itemize}
\item[@] Third item.
\end{itemize}

\section{Comments}
Believe that life is worth living % Note comic irony in the very first sentence
, and your belief will help create the fact.

% This is a comment.

\vspace{12pt}
\vspace{12pt}
\vspace{12pt}
\vspace{12pt}
\vspace{12pt}
This is spacing.

\# \$ \% \^{} \& \_ \{ \} \~{}

\section{Spacing}
\vspace{12pt}
\vspace{12pt}
\vspace{12pt}

Item \#1A\textbackslash642 costs \$8 \& is sold at a \~{}10\% profit.


\stepcounter{chapter} % if we do \stepcounter{chapter}{3} it will produce extra 3
\addcontentsline{toc}{chapter}{Practical 3: Tables}
\chapter*{Practical 3: Tables}

\section{List of Tables}

\begin{table}[ht]
\label{Table 1}
	\begin{tabular}{|l|l|}
	Apples & Green\\
	Strawberries & Red\\
	Oranges & Orange\\
	\end{tabular}
\caption{Example Table 1}
\label{Table 1}
\end{table}

\begin{table}[ht]
\label{Table 2}
	\begin{tabular}{rc}
	Appels & Green\\
	\hline
	Strawberries & Red\\
	\cline{2-2}
	Oranges & Orange\\
	\end{tabular}
\caption{Example Table 2}
\end{table}

\begin{table}[ht]
\label{Table 3}
\begin{tabular}{|r|l|}
	\hline
	8 & here's\\
	\cline{2-2}
	86 & stuff\\
	\hline
	\hline
	2008 & now\\
	\hline
	\end{tabular}
\caption{Example Table 3}
\end{table}



\begin{tabular}{|c|c|c|c|}
\hline
\multicolumn{4}{|c|}{Country List}\\
\hline
Country Name & ALPHA 2 Code & ALPHA 3 Code & Numeric Code\\
\hline
Afghanistan & AF & AFG & 004\\
Albania & AL & ALB & 008\\
Algeria & DZ & DZA & 012\\
Angola & AO & AGO & 024\\
\hline
\end{tabular}

\newpage

\begin{tabular}{l|r|r}
Item & Quantity & Price (\$)\\
\hline
Nails & 500 & 0.34\\
Wooden boards & 100 & 4.00\\
Bricks & 240 & 11.50\\
\end{tabular}

\begin{tabular}{c|ccc}
	& \multicolumn{3}{c}{Year}\\
\cline{2-4}
City & 2006 & 2007 & 2008\\
\hline
London & 45789 & 46551 & 51298\\
Berlin & 34549 & 32543 & 29870\\
Paris & 49835 & 51009 & 51970\\
\end{tabular}

\stepcounter{chapter} % if we do \stepcounter{chapter}{4} it will produce extra 4
\addcontentsline{toc}{chapter}{Practical 4: Figures and Equations}
\chapter*{Practical 4: Figures and Equations}

\section{Figures}

\begin{figure}[h]
\centering
\includegraphics[width=0.25\textwidth]{1}
\caption{One}
\label{image-1} % need in reference
\end{figure}

Figure \ref{image-1} shows my histogram.

\begin{figure}[h]
\centering
\includegraphics[width=0.25\textwidth]{2}
\caption{Two}
\end{figure}

\section{Sub Figures}

\begin{figure}[h!]
\begin{subfigure}{0.4\textwidth}
\centering
\includegraphics[width=0.25\textwidth]{1}
\caption{One}
\label{fig:subimage1}
\end{subfigure}
\begin{subfigure}{0.4\textwidth}
\centering
\includegraphics[width=0.25\textwidth]{2}
\caption{Two}
\label{fig:subimage2}
\end{subfigure}

\caption{subimage1 and subimage2}
\label{fig:image2}
\end{figure}

\section{Equations}
$1+2=3$
$$1+2=3$$	% center of the page
\begin{equation}
1+2=3
\end{equation}

\begin{eqnarray}
a&=&b+c \\
&=&y-z \\
\end{eqnarray}

$n^2$ \\
$2_a$ \\
$b_{a-2}$ \\
$$\frac{a}{3}$$ \\
$$\frac{y}{\frac{3}{x}+b}$$ \\
$$\sqrt{y^2}$$ \\
$$\sqrt[x]{y^2}$$ \\
$$\sum_{x=1}^5 y^z$$ \\
$$\lim_{x \to \infty} f(x)$$ \\
$$\int_a^b f(x)$$ \\

\begin{equation*}
	\left[
	\begin{matrix}
	1 & 0\\
	0 & 1\\
	\end{matrix}
	\right]
\end{equation*}

$\alpha$ \\
$\beta$ \\
$\delta,\Delta$ \\
$\theta,\Theta$ \\
$\mu$ \\
$\pi,\Pi$ \\
$\sigma,\Sigma$ \\
$\phi,\Phi$ \\
$\psi,\Psi$ \\
$\omega,\Omega$ \\

\begin{eqnarray}
e = mc^2 \\
\pi = \frac{c}{d} \\
\frac{d}{dx} e^x = e^x \\
\frac{d}{dx} \int_0^\infty f(s)ds = f(x) \\
f(x) = \sum_{x=i} = 0^\infty\frac{f^{(i)}{(0)}}{i!}x^i \\
x = \sqrt{\frac{x_i}{z}y} \\
\end{eqnarray}

\begin{equation*}
	\left[
	\begin{matrix}
	1 & 2 & 3 & 4 & 5\\
	6 & 7 & 8 & 9 & 10\\
	11 &12 & 13 & 14 & 15\\
	16 & 17 & 18 & 19 & 20\\
	21 & 22 & 23 & 24 & 25\\
	\end{matrix}
	\right]
\end{equation*}

\newpage

\stepcounter{chapter} % if we do \stepcounter{chapter}{3} it will produce extra 3
\addcontentsline{toc}{chapter}{Practical 5: References}
\chapter*{Practical 5: References}

\section{The BibTex file}

\section{Citing references}

\cite{Birdetal2001} This is a normal reference of the Birdetal2001 paper. \\
\cite[p. 215]{Birdetal2001} This is a demo page reference of the Birdetal2001 paper. \\
\cite{9654185, 9001082, w1, p1, p2, p3, p4, p5, p6} This is a demo multiple reference. \\
\cite{9654185} This is just a demo citation of the 1st bib reference. \\
This is just a demo citation of the 2nd bib reference \cite{9001082}. \\

\section{Styles}
\subsection{Numerical citations}

\bibliographystyle{plain}
\bibliography{references}


\end{document}